% (The MIT License)
%
% Copyright (c) 2021 Yegor Bugayenko
%
% Permission is hereby granted, free of charge, to any person obtaining a copy
% of this software and associated documentation files (the 'Software'), to deal
% in the Software without restriction, including without limitation the rights
% to use, copy, modify, merge, publish, distribute, sublicense, and/or sell
% copies of the Software, and to permit persons to whom the Software is
% furnished to do so, subject to the following conditions:
%
% The above copyright notice and this permission notice shall be included in all
% copies or substantial portions of the Software.
%
% THE SOFTWARE IS PROVIDED 'AS IS', WITHOUT WARRANTY OF ANY KIND, EXPRESS OR
% IMPLIED, INCLUDING BUT NOT LIMITED TO THE WARRANTIES OF MERCHANTABILITY,
% FITNESS FOR A PARTICULAR PURPOSE AND NONINFRINGEMENT. IN NO EVENT SHALL THE
% AUTHORS OR COPYRIGHT HOLDERS BE LIABLE FOR ANY CLAIM, DAMAGES OR OTHER
% LIABILITY, WHETHER IN AN ACTION OF CONTRACT, TORT OR OTHERWISE, ARISING FROM,
% OUT OF OR IN CONNECTION WITH THE SOFTWARE OR THE USE OR OTHER DEALINGS IN THE
% SOFTWARE.

\documentclass[12pt]{article}
\usepackage[tt=false,type1=true]{libertine}
\usepackage{multicol}
\usepackage{tbd}
\usepackage{ffcode}
\title{\ff{tbd}: \LaTeX{} Package \\ for highlighting places that require more work}
\author{Yegor Bugayenko}
\date{0.0.0 00.00.0000}
\begin{document}
\pagenumbering{gobble}
\raggedbottom
\setlength{\parindent}{0pt}
\setlength{\columnsep}{32pt}
\setlength{\parskip}{6pt}
\maketitle

This package helps you highlight places in your articles
and make sure it looks nice. Install it from CTAN and then
use like this:

\begin{multicols}{2}
\setlength{\parskip}{0pt}
\scriptsize
\raggedcolumns
\begin{verbatim}
\documentclass{article}
\usepackage{tbd}
\begin{document}
The budget is \tbd{99.00 USD}.
\end{document}
\end{verbatim}

\columnbreak

The budget is \tbd{99.00 USD}.

\end{multicols}

With this package it's also possible to highlight
\tbd{larger pieces of text that take \textbf{a few lines}} on the page.

If you want to hide the content of what's inside \ff{\char`\\tbd},
use \ff{hide} option of the package. They will all be replaced with
\tbd{TBD} placeholders.

If you want them to disappear entirely, use the \ff{off} package option.

More details about this package you can find
in the \ff{yegor256/tbd} GitHub repository.

\end{document}